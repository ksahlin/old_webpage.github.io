%%%%%%%%%%%%%%%%%%%%%%%%%%%%%%%%%%%%%%%%%
% Medium Length Professional CV
% LaTeX Template
% Version 2.0 (8/5/13)
%
% This template has been downloaded from:
% http://www.LaTeXTemplates.com
%
% Original author:
% Trey Hunner (http://www.treyhunner.com/)
%
% Important note:
% This template requires the resume.cls file to be in the same directory as the
% .tex file. The resume.cls file provides the resume style used for structuring the
% document.
%
%%%%%%%%%%%%%%%%%%%%%%%%%%%%%%%%%%%%%%%%%

%----------------------------------------------------------------------------------------
%   PACKAGES AND OTHER DOCUMENT CONFIGURATIONS
%----------------------------------------------------------------------------------------

\documentclass{resume} % Use the custom resume.cls style

\usepackage[left=0.75in,top=0.6in,right=0.75in,bottom=0.6in]{geometry} % Document margins
\usepackage{hyperref}
\usepackage{amsmath}
\renewcommand{\refname}{Peer reviewed publications}
\name{Kristoffer Sahlin, PhD} % Your name
\address{UPDATED: Dec 2, 2018} % Your address
% \address{Pontonj{\"a}rgatan 18 \\ Stockholm, Sweden} % Your address
% \address{+8147778944 \\ krsahlin@gmail.com} % Your phone number and email
\begin{document}

%----------------------------------------------------------------------------------------
%   PERSONAL INFORMATION SECTION
%----------------------------------------------------------------------------------------

\begin{rSection}{Contact information}

Tel: +1 814 777 8944\\
Email: kxs624@psu.edu / krsahlin@gmail.com\\
Web: \url{http://ksahlin.github.io/}
\end{rSection}

%----------------------------------------------------------------------------------------
%   EDUCATION SECTION
%----------------------------------------------------------------------------------------

\begin{rSection}{Education \& Training}
{\bf Postdoctoral researcher} \hfill {\em Oct 2015 - } \\ 
University: Pennsylvania State University \\
Advisor: Associate Professor Paul Medvedev\\
{\bf Ph.D. in Computer Science} \hfill {\em Sept 2010 - Sept 2015} \\ 
University: Royal Institute of Technology (KTH), Sweden \\
Thesis: Algorithms and statistical models for scaffolding contig assemblies and detecting structural variants using read pair data \\
Advisor: Associate Professor Lars Arvestad\\
Co-advisor: Professor Joakim Lundeberg\\
{\bf M.Sc. in Mathematical Statistics} \hfill {\em Aug 2008 - Sept 2010} \\ 
University: Stockholm University, Sweden \\
Thesis: Estimating convergence of Markov chain 
Monte Carlo simulations\\
Advisor: Assistant Professor Sebastial H\"{o}hna\\
{\bf B.S. in Mathematics} \hfill {\em Aug 2005 - June 2008} \\ 
University: Stockholm University, Sweden \\
Thesis: Splines: A theoretical and computational study\\
Advisor: Professor Hans Rullg\r{a}rd\\
\end{rSection}

% %----------------------------------------------------------------------------------------
% % RESEARCH INTERESTS
% %----------------------------------------------------------------------------------------


% \begin{rSection}{Research interests}
% Transcriptomics and genomics. 
% \end{rSection}

%----------------------------------------------------------------------------------------
%   VISITING RESEARCH EXPERIENCE SECTION
%----------------------------------------------------------------------------------------

\begin{rSection}{Visiting research experience}

\begin{rSubsection}{Helsinki University}{September 2014}{Visiting researcher}{Helsinki, Finland}
\item Visiting Veli M{\"a}kinen's lab for work on scaffolding and gapfilling of genome assemblies.
\end{rSubsection}

\begin{rSubsection}{Penn State University}{November 2014}{Visiting researcher}{State college, PA, USA}
\item Visiting Paul Medvedev's lab for work on Structural variation detection.
\end{rSubsection}
\end{rSection}
%----------------------------------------------------------------------------------------
%   PRESENTATIONS SECTION
%----------------------------------------------------------------------------------------

\begin{rSection}{Presentations}
\begin{itemize}
    \item ISMB, HitSeq track (2018)
    \item Genome Informatics (2017)
    \item RECOMB (2016)
    \item WABI (2015)
\end{itemize}
\end{rSection}

%----------------------------------------------------------------------------------------
%   AWARDS
%----------------------------------------------------------------------------------------

\begin{rSection}{Awards and grants}
\begin{itemize}
\item KTH opportunities fund: Investing in research talent grant, 2014.
\item Grant proposal finalist in the PacBio 2018 Iso-Seq SMRT Grant Program.
\end{itemize}

\end{rSection}

%----------------------------------------------------------------------------------------
%   RESEARCH AREAS SECTION
%----------------------------------------------------------------------------------------

% \begin{rSection}{Research areas}

% Genome assembly, Structural variation detection, Transcriptome assembly and analysis, Phylogenetics. 

% \end{rSection}

%----------------------------------------------------------------------------------------
%   OTHER INTERESTS
%----------------------------------------------------------------------------------------

% \begin{rSection}{PROFESSIONAL INTERESTS}

% Statistical modeling and inference, stochastic processes, graph theory, combinatorics, probability theory

% \end{rSection}


%----------------------------------------------------------------------------------------
%   TEACHING SECTION
%----------------------------------------------------------------------------------------

\begin{rSection}{Teaching}
\begin{itemize}
    \item Teaching assistant in Statistical Methods in Applied Computer Science at KTH (2012, 2013, 2014).
    \item Teaching assistant and assisting lecturer in graduate course Applied bioinformatics at KTH. I gave five lectures on introduction of python. TA throughout the course (2013).
    \item Teaching assistant and recitation session lecturer in courses: Programming Techniques and Matlab (Undergraduate level), Programming Techniques and C (Undergraduate level), and Bioinformatics and Biostatistics (graduate level), KTH (2013).
\end{itemize}

% \begin{rSubsection}{2014}{}{}{}
% \item TA in Statistical Methods in Applied Computer Science at KTH Royal Institute of Technology, Stockholm, Sweden.
% \end{rSubsection}

% \begin{rSubsection}{2013}{}{}{}
% \item Assisting lecturer in course Applied bioinformatics at KTH Royal Institute of Technology, Stockholm, Sweden. I gave five lectures on introduction of python. TA throughout the course. 
% \item TA (Lecturer at exercise sessions and computer lab assistant) in courses: Programming Techniques and Matlab, Programming Techniques and C, Bioinformatics and Biostatistics, and Statistical Methods in Applied Computer Science at KTH Royal Institute of Technology, Stockholm, Sweden.
% \end{rSubsection}

% \begin{rSubsection}{2012}{}{}{}
% \item TA in courses: Algorithmic bioinformatics, Applied bioinformatics, and Statistical Methods in Applied Computer Science at KTH Royal Institute of Technology, Stockholm, Sweden. 
% \end{rSubsection}

\end{rSection}
%----------------------------------------------------------------------------------------
%   STUDENT SUPERVISION SECTION
%----------------------------------------------------------------------------------------

\begin{rSection}{Advising}
% \begin{rSubsection}{}{March - June, 2014.}{}{}
\item Josefine R{\"o}hss - Analysing k-mer distributions in a genome sequencing project. Bachelor's Thesis, March - June, 2014.
% \end{rSubsection}
\end{rSection}

%----------------------------------------------------------------------------------------
%   EXPERIENCE SECTION
%----------------------------------------------------------------------------------------

% \begin{rSection}{Experience}

% \end{rSection}



%----------------------------------------------------------------------------------------
%   ACADEMIC SERVICE SECTION
%----------------------------------------------------------------------------------------

\begin{rSection}{Academic service}

\begin{itemize}
\item Reviewer for journals: BMC Bioinformatics, Bioinformatics, GigaScience, Communications in Statistics - Simulation and Computation.
\end{itemize}

\begin{itemize}
\item Reviewer for conferences: ISMB (2017, 2018), RECOMB (2014, 2016 - 2019), RECOMB-seq (2018) WABI 2015 
\end{itemize}

\end{rSection}

%----------------------------------------------------------------------------------------
%   Publications
%----------------------------------------------------------------------------------------

%\begin{rSection}{Publications}
%\begin{itemize}
%\item See google Scholar \url{http://scholar.google.com/citations?user=IAGyYNEAAAAJ&hl=en}
%\end{itemize}
%
%\end{rSection}

%----------------------------------------------------------------------------------------
%   TECHNICAL STRENGTHS SECTION
%----------------------------------------------------------------------------------------

% \begin{rSection}{Technical Strengths}

% \begin{tabular}{ @{} >{\bfseries}l @{\hspace{6ex}} l }
% Computer Languages & Python (Advanced), R, MatLab, C/C++ (basic - intermediate)\\
% Tools & GitHub, Vi/Vim, Unix environment, LaTex, Snakemake
% \end{tabular}

% \end{rSection}

%----------------------------------------------------------------------------------------
%   EXAMPLE SECTION
%----------------------------------------------------------------------------------------

\nocite{*}
%\begin{rSection}{Section Name}

%Section content\ldots

%\end{rSection}


%----------------------------------------------------------------------------------------
% \bibliographystyle{plain}
\bibliographystyle{unsrt}
%\bibliographystyle{natbib}
%\bibliographystyle{apalike}
\bibliography{cv_bib}

\end{document}
