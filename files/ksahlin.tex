%%%%%%%%%%%%%%%%%%%%%%%%%%%%%%%%%%%%%%%%%
% Medium Length Professional CV
% LaTeX Template
% Version 2.0 (8/5/13)
%
% This template has been downloaded from:
% http://www.LaTeXTemplates.com
%
% Original author:
% Trey Hunner (http://www.treyhunner.com/)
%
% Important note:
% This template requires the resume.cls file to be in the same directory as the
% .tex file. The resume.cls file provides the resume style used for structuring the
% document.
%
%%%%%%%%%%%%%%%%%%%%%%%%%%%%%%%%%%%%%%%%%

%----------------------------------------------------------------------------------------
%	PACKAGES AND OTHER DOCUMENT CONFIGURATIONS
%----------------------------------------------------------------------------------------

\documentclass{resume} % Use the custom resume.cls style

\usepackage[left=0.75in,top=0.6in,right=0.75in,bottom=0.6in]{geometry} % Document margins
\usepackage{hyperref}
\usepackage{amsmath}
\renewcommand{\refname}{Selected publications}
\name{Kristoffer Sahlin} % Your name
\address{UPDATED:  May 8th, 2018} % Your address
% \address{Pontonj{\"a}rgatan 18 \\ Stockholm, Sweden} % Your address
\address{+8147778944 \\ krsahlin@gmail.com} % Your phone number and email

\begin{document}

%----------------------------------------------------------------------------------------
%   PERSONAL INFORMATION SECTION
%----------------------------------------------------------------------------------------

\begin{rSection}{Personal information}

Date of birth: March 05, 1984 Stockholm, Sweden\\
Nationality: Swedish\\
Website: \url{http://ksahlin.github.io/}\\
\end{rSection}

%----------------------------------------------------------------------------------------
%	EDUCATION SECTION
%----------------------------------------------------------------------------------------

\begin{rSection}{Education}

{\bf Ph.D. in Computer Science} \hfill {\em Sept 2010 - Sept 2015} \\ 
School: Royal institute of Technology (KTH) \\
Thesis: Algorithms and statistical models for scaffolding contig assemblies and detecting structural variants using read pair data \\
Advisor: Associate Professor Lars Arvestad\\
Co-advisor: Professor Joakim Lundeberg\\
{\bf M.Sc. in Mathematical Statistics} \hfill {\em Aug 2008 - Sept 2010} \\ 
School: Stockholm University \\
Thesis: Estimating convergence of Markov chain 
Monte Carlo simulations\\
Advisor: Sebastial H\"{o}hna\\
{\bf B.S. in Mathematics} \hfill {\em Aug 2005 - June 2008} \\ 
School: Stockholm University \\
Thesis: Splines: A theoretical and computational study\\
Advisor: Hans Rullg\r{a}rd\\
\end{rSection}

%----------------------------------------------------------------------------------------
%   PROFESSIONAL EXPERIENCE SECTION
%----------------------------------------------------------------------------------------

\begin{rSection}{Professional experience}

{\bf Postdoctoral researcher} \hfill {\em Oct 2015 - } \\ 
School: Penn State University \\
Advisor: Associate Professor Paul Medvedev
\end{rSection}


%----------------------------------------------------------------------------------------
%	RESEARCH AREAS SECTION
%----------------------------------------------------------------------------------------

\begin{rSection}{Research areas}

Genome assembly, Structural variation detection, Transcriptome assembly and analysis, Phylogenetics. 

\end{rSection}

%----------------------------------------------------------------------------------------
%	OTHER INTERESTS
%----------------------------------------------------------------------------------------

\begin{rSection}{PROFESSIONAL INTERESTS}

Statistical modeling and inference, stochastic processes, graph theory, combinatorics, probability theory

\end{rSection}


%----------------------------------------------------------------------------------------
%   TEACHING SECTION
%----------------------------------------------------------------------------------------

\begin{rSection}{Teaching}

\begin{rSubsection}{2014}{}{}{}
\item Teaching assistant (correcting homework assignments, general supervision) in Statistical Methods in Applied Computer Science at KTH Royal Institute of Technology, Stockholm, Sweden.
\end{rSubsection}

\begin{rSubsection}{2013}{}{}{}
\item Lecturer in course Applied Bioinformatics at KTH Royal Institute of Technology, Stockholm, Sweden. I taught five lectures on introduction of python. Teachers assistant throughout the course. 
\item Teaching assistant (Lecturer at exercise sessions and computer lab assistant) in Programming Techniques and Matlab at KTH Royal Institute of Technology, Stockholm, Sweden
\item Teaching assistant (Lecturer at exercise sessions and computer lab assistant) for Programming Techniques and C at KTH Royal Institute of Technology, Stockholm, Sweden
\item Teaching assistant (for computer labs) for Bioinformatics and Biostatistics at KTH Royal Institute of Technology, Stockholm, Sweden
\item Teaching assistant (correcting homework assignments, general supervision) for Statistical Methods in Applied Computer Science at KTH Royal Institute of Technology, Stockholm, Sweden.
\end{rSubsection}

\begin{rSubsection}{2012}{}{}{}
\item Teaching assistant for course Algorithmic bioinformatics at KTH Royal Institute of Technology, Stockholm, Sweden. 
\item Teaching assistant for course Applied bioinformatics at KTH Royal Institute of Technology, Stockholm, Sweden.
\item Teaching assistant (correcting homework assignments, general supervision) for Statistical Methods in Applied Computer Science at KTH Royal Institute of Technology, Stockholm, Sweden.
\end{rSubsection}

\end{rSection}
%----------------------------------------------------------------------------------------
%   STUDENT SUPERVISION SECTION
%----------------------------------------------------------------------------------------

\begin{rSection}{Student supervision}
% \begin{rSubsection}{}{March - June, 2014.}{}{}
\item Josefine R{\"o}hss - Analysing k-mer distributions in a genome sequencing project. Bachelor's Thesis, March - June, 2014.
% \end{rSubsection}
\end{rSection}

%----------------------------------------------------------------------------------------
%	EXPERIENCE SECTION
%----------------------------------------------------------------------------------------

\begin{rSection}{Experience}

\begin{rSubsection}{Helsinki University}{September 2014}{Visiting researcher}{Helsinki, Finland}
\item Visiting Veli M{\"a}kinen's lab for work on scaffolding and gapfilling of genome assemblies.
\end{rSubsection}

\begin{rSubsection}{Penn State University}{November 2014}{Visiting researcher}{State college, PA, USA}
\item Visiting Paul Medvedev's lab for work on Structural variation detection.
\end{rSubsection}

%------------------------------------------------

\begin{rSubsection}{Conferences/Meetings}{}{}{}
\item ISMB: 2012, 2013, 2014.
\item RECOMB: 2016 \textbf{(Speaker)}.
\item Genome informatics: 2013, 2014, 2017 \textbf{(Speaker)}.
\item WABI: 2015 \textbf{(Speaker)}.
\item Assemblathon 1 satellite meeting for Genome informatics 2011.
\item GATC Plant genomics symposium (2012). 
\end{rSubsection}

%------------------------------------------------

\end{rSection}

%----------------------------------------------------------------------------------------
%	AWARDS
%----------------------------------------------------------------------------------------

\begin{rSection}{Awards and grants}
\begin{itemize}
\item KTH opportunities fund, Investing in research talent grant, 2014.
\end{itemize}

\end{rSection}

%----------------------------------------------------------------------------------------
%   ACADEMIC SERVICE SECTION
%----------------------------------------------------------------------------------------

\begin{rSection}{Academic service}

\begin{itemize}
\item Reviewer for journals: BMC Bioinformatics, Bioinformatics, GigaScience, Communications in statistics - Simulation and Computation.
\end{itemize}

\begin{itemize}
\item Reviewer for conferences: ISMB (2017, 2018), RECOMB (2014, 2016, 2017, 2018), WABI 2015 
\end{itemize}

\end{rSection}

%----------------------------------------------------------------------------------------
%	Publications
%----------------------------------------------------------------------------------------

%\begin{rSection}{Publications}
%\begin{itemize}
%\item See google Scholar \url{http://scholar.google.com/citations?user=IAGyYNEAAAAJ&hl=en}
%\end{itemize}
%
%\end{rSection}

%----------------------------------------------------------------------------------------
%	TECHNICAL STRENGTHS SECTION
%----------------------------------------------------------------------------------------

\begin{rSection}{Technical Strengths}

\begin{tabular}{ @{} >{\bfseries}l @{\hspace{6ex}} l }
Computer Languages & Python (Advanced), R, MatLab, C/C++ (basic - intermediate)\\
Tools & GitHub, Vi/Vim, Unix environment, LaTex, Snakemake
\end{tabular}

\end{rSection}

%----------------------------------------------------------------------------------------
%	EXAMPLE SECTION
%----------------------------------------------------------------------------------------

\nocite{*}
%\begin{rSection}{Section Name}

%Section content\ldots

%\end{rSection}


%----------------------------------------------------------------------------------------
% \bibliographystyle{plain}
\bibliographystyle{unsrt}
%\bibliographystyle{natbib}
%\bibliographystyle{apalike}
\bibliography{cv_bib}

\end{document}
